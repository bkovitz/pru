\documentclass[letterpaper,11pt,fleqn]{article}

\usepackage{listings}
\lstset{basicstyle=\footnotesize\ttfamily}

\usepackage{enumitem}
\usepackage{tabulary}
\usepackage{siunitx}

\usepackage{hyperref}
\usepackage{url}
\urlstyle{sf}

\newcommand{\techref}{\textit{AM335x Technical Reference Manual}}

\newenvironment{docref}
   {\vspace{\baselineskip}\noindent\begin{minipage}{\textwidth}\raggedright}
   {\end{minipage}}

\title{Real-Time Programming with the PRU \\ on the Beaglebone}
\author{Ben Kovitz}
\date{September 2015}

\begin{document}
\maketitle

\tableofcontents

\section{Introduction}

This document shows you how to use the the PRU (Programmable Real-time Unit)
of the AM3358 Sitara processor on a BeagleBone Black to do real-time
programming.

The example code in this document generates and captures PWM (Pulse-Width
Modulation): one PRU generates pulses while the other PRU captures pulses.
Code running concurrently on the ARM processor sets outgoing pulse widths and
reads measurements of incoming pulse widths through memory shared with the
PRU.  This is probably not the best way to do PWM on an AM3358: the AM3358 has
a built-in PWM unit just for this kind of task, so the PRU really isn't
necessary. However, using the PRU to perform PWM does illustrate real-time
programming on a very simple but highly timing-sensitive example. It should be
straightforward to modify the examples presented here to make the PRU perform
tasks with hard real-time constraints of about 24~MHz while the ARM runs Linux
concurrently.

\subsection{Must-have documentation}

These are the authoritative manuals from Texas Instruments:

\nopagebreak
\begin{docref}
\techref \\
\url{http://www.ti.com/lit/ug/spruh73l/spruh73l.pdf} \\
4,972 pages, including all memory maps, but it still doesn't explain everything.
\end{docref}

\begin{docref}
\textit{AM335x PRU-ICSS Reference Guide} \\
\url{https://github.com/beagleboard/am335x\_pru\_package/blob/master/am335xPruReferenceGuide.pdf} \\
Includes complete documentation of the PRU instruction set.
\end{docref}

\begin{docref}
\textit{PRU Assembly Instructions} \\
\url{http://processors.wiki.ti.com/index.php/PRU\_Assembly\_Instructions} \\
A more-convenient description of the PRU instruction set.
\end{docref}

\begin{docref}
\textit{PRU Linux Application Loader API Guide} \\
\url{http://processors.wiki.ti.com/index.php/PRU\_Linux\_Application\_Loader\_API\_Guide} \\
The C API for loading programs into PRU memory, enabling and disabling the
PRUs, etc.
\end{docref}

\vspace{\baselineskip}
\noindent
For hooking up hardware, you'll also need BeagleBoard's
documentation:

\begin{docref}
\textit{BeagleBone Rev A6 System Reference Manual} \\
\url{http://beagleboard.org/static/beaglebone/latest/Docs/Hardware/BONE\_SRM.pdf}
\end{docref}

\vspace{\baselineskip}
\noindent
The pinouts of the P8 and P9 expansion boards are explained on pp.~54--63.
Convenient spreadsheet forms of the pinouts are here:
\url{http://www.embedded-things.com/bbb/beaglebone-black-pin-mux-spreadsheet/}.

\subsection{PRU background}

The PRU is itself a microprocessor with a relatively simple instruction set.
The AM335x contains two PRUs in addition to an ARM Cortex-A8 microprocessor.
Each PRU has 32 registers, each 32 bits wide, and each instruction is 32 bits
wide.

The PRU runs at 200~MHz, has four buses, and no pipeline. Most
instructions take only one clock cycle despite the lack of a pipeline. Since
there is no pipeline, calculating the time required to execute PRU code is
often as simple as counting the instructions and multiplying by
\SI{5}{\nano\second}. The PRU even has a single-clock-cycle integer multiply
instruction.

The AM335x has four memory spaces:

{
\renewcommand{\arraystretch}{1.5}
\small
\nopagebreak
\vspace{\baselineskip}
\begin{tabulary}{\dimexpr\textwidth-\parindent\relax}{LLLJ}
level & address space & min. latency & \\
\hline
L1~\begin{minipage}[t]{1.5in}Instruction \& \\ Data Cache\end{minipage}
& 32K & single-cycle & Includes PRU instructions and scratchpad registers. \\
L2 Cache & 256K & \SI{8}{\nano\second}, \SI{20}{\nano\second} if cache miss
& Includes ARM/PRU shared memory. \\
L3 & Full 32-bit addressing & \SI{40}{\nano\second} & Access to DDR memory. \\
L4 & Full 32-bit addressing & $>$\SI{40}{\nano\second}? & Access to peripherals and GPIO ports. \\
\end{tabulary}
}

\vspace{\baselineskip}
As long as a PRU instruction is accessing L1 memory, it can read the
instruction, read the data, and write the result in a single clock cycle.
Accessing L2 memory introduces a delay of two to five clock cycles
plus additional uncertainty due to possible contention with the ARM. L3 and L4
introduce delays of at least 8 clock cycles and additional nondeterminism
because of possible contention with other devices on the AM335x.

% I never found a completely authoritative source for the above numbers. BEN 

Fortunately, the PRU provides ways to reduce or avoid accesses to memory
beyond L1. The scratchpad registers R0--R29 can be loaded at the start of a
program and serve as memory; PRU registers can even perform indirection,
providing the ``address'' of an operand stored in another register.
Bits in registers R30 and R31 of the PRU can be configured for direct
access to GPIO bits, reducing access time to a single PRU clock cycle. (See
section~4.4.1.2.3 of the \techref.)
The PRU has a set of hard-coded ``constant'' registers, C0--C31, which hold
frequently used addresses so these need not occupy the scratchpad registers.
(See section~4.4.1.1 of the \techref.) Many addresses in the L2 and L3
spaces lead to the same memory, so the PRU can avoid latency by accessing them
through the L2 addresses. In particular, this is the best way for the PRU to
access shared memory for communicating with the ARM.

The PRU does not support interrupts. Interrupts would complicate real-time
processing and make response times unpredictable. Instead, the PRU must poll
sources of interrupts. The documentation still calls them ``interrupts'' and 
the PRU has an ``interrupt controller'' (INTC) to track and prioritize them.

There is no C compiler for the PRU. So, you must write all PRU code in
assembly language by hand. The assembler for the PRU is called \textit{pasm}.

In some documentation, the PRU is also called the PRUSS (Programmable
Real-Time Unit Subsystem) and the \mbox{PRU-ICSS} (Programmable Real-Time Unit
and Industrial Communication Subsystem).

\subsection{Installing and compiling}

I compiled all the code in this document on the BeagleBone Black itself under
Debian~7. Running the following commands as the superuser on the BeagleBone
Black will install the necessary development tools:

\begin{lstlisting}{language=bash}
  # apt-get install gdb
  # apt-get install gpsd gpsd-clients python-gps libgps-dev
\end{lstlisting}

@@Various stuff goes here.

\section{Generating PWM}

@@various stuff.

\section{Capturing PWM}

@@various stuff.

\section{GPS}

@@various stuff

\end{document}
